  %
  % Sjabloon voor de master ingenieurswetenschappen: computerwetenschappen
  %
  \documentclass[master=cws, masteroption=mmc]{kulemt}
  \usepackage[dutch]{babel}
  %%%%%%%%%%%%%% Wijzig niets boven deze regel %%%%%%%%%%%%%

  % Vul de titel van jouw masterproef hieronder in tussen { en }.
  \setup{title={Beste masterproef ooit al geschreven},
  %
  % Vul hieronder namen in, steeds Voornaam Naam.
  % Indien meerdere auteurs, assessoren, assistenten, scheidt hun namen
  % met \and .
  author={Een Auteur\and Tweede Auteur},
  promotor={Prof.\,dr.\,ir.\ Weet Beter},
  assessor={Ir.\,W. Eetveel\and W. Eetrest},
  assistant={Ir.\ A.~Assistent \and D.~Vriend}}
  % 
  \setup{filingcard, % Deze regel niet wijzigen
  %
  % Vul de vertaalde titel van jouw masterproef hieronder in tussen { en }.
  translatedtitle={The best master thesis ever},
  %
  % UDC nummer is richtingafhankelijk. zie http://www.udcc.org/udcsummary/php/index.php?id=36946&lang=en voor het UDC nummer.
  udc=621.3,
  %
  % Hieronder, tussen { en } een korte samenvatting toevoegen.
  % Lege regels en het commando \par zijn niet toegelaten.
  % Wees voorzichtig met speciale TeX-tekens #$%&^_~{}\ !!
  shortabstract={%
  Hier komt een heel bondig abstract van hooguit 500
  woorden. \LaTeX\ commando's mogen hier gebruikt worden.
  Voorbeelden van letters met accenten zijn 
  ``{\'e}{\"\i}{\c{c}}{\`a}{\^o}''.
  (Gebruik {\"\i} in plaats van {\"i}, om geen 3 puntjes te hebben.)
  }}

  %%%%%%%%% Wijzig niets onder deze regel %%%%%%%%
  %\setup{coverpageonly}



  % Choose the main text font (e.g., Latin Modern)
  \setup{font=lm}

  % If you want to include other LaTeX packages, do it here.

  % Finally the hyperref package is used for pdf files.
  % This can be commented out for printed versions.
  \usepackage[pdfusetitle,colorlinks,plainpages=false]{hyperref}



  \begin{document}
    \begin{preface}[The Author\\ \textup{1 January 2010}]
      The text of the preface. A few paragraphs should follow.
    \end{preface}
    
    \tableofcontents*
    
    \begin{abstract}
      abstract
    \end{abstract}
    
    % A list of figures and tables is optional
    %\listoffigures
    %\listoftables
    % If you only have a few figures and tables you can use the following instead
    %\listoffiguresandtables
    % The list of symbols is also optional.
    % This list must be created manually, e.g., as follows:
%     \chapter{List of Abbreviations and Symbols}
%     \section*{Abbreviations}
%     \begin{flushleft}
%     \renewcommand{\arraystretch}{1.1}
%     \begin{tabularx}{\textwidth}{@{}p{12mm}X@{}}
%     LoG
%     & Laplacian-of-Gaussian \\
%     MSE
%     & Mean Square error \\
%     PSNR & Peak Signal-to-Noise ratio \\
%     \end{tabularx}
%     \end{flushleft}
%     \section*{Symbols}
%     \begin{flushleft}
%     \renewcommand{\arraystretch}{1.1}
%     \begin{tabularx}{\textwidth}{@{}p{12mm}X@{}}
%     42
%     & ‘‘The Answer to the Ultimate Question of Life, the Universe,
%     and Everything’’ according to \cite{h2g2} \\
%     $c$
%     & Speed of light \\
%     $E$
%     & Energy \\
%     $m$
%     & Mass \\
%     $\pi$ & The number pi \\
%     \end{tabularx}
%     \end{flushleft}

    % Now comes the main text
    \mainmatter



    
    \chapter{Inleiding}
      In dit hoofdstuk wordt een algemene inleiding gegeven waarin het probleem gekaderd wordt, 
      monte carlo integratie wordt kort uitgelegd aan de hand van de vergelijking, 
      hierdoor wordt het probleem duidelijk.
      \par
      [Figuur die een scene toont gerenderd met monte carlo rendering, met een klein aantal samples]
      \par
      [Figuur die een scene toont gerenderd met monte carlo rendering, met een groot aantal samples]
    
    \chapter{Gerelateerd werk}
      In dit hoofdstuk worden papers met gerelateerde inhoud besproken en met het RPF algoritme vergeleken, 
      belangrijk is om verbanden te leggen tussen deze papers en ze zo op te delen in de verschillende stromingen waar ze deel van uitmaken.
      \par
      (figuren? of gewoon droge tekst?)
      
    \chapter{Bilateral filter}
      In dit hoofdstuk wordt uitgelegd hoe de bilateral filter werkt aangezien het RPF algoritme hierop gebaseerd is.
      Net zoals in de paper volgt hier eerst de uitleg over een simpele Gaussiaanse filter, 
      gevolgd door de gewone bilateral filter gevolgd door de cross bilateral filter.
      \par
      [Figuur die gefilterd is door een Gaussiaanse filter]
      \par
      [Figuur die gefilterd is door een bilateral filter]
      \par
      ([Figuur die alleen gefilterd is door een cross bilateral filter] -> dit is al rpf, dus niet meer toevoegen (verwijzen naar eerste figuren bovenaan de thesis)?)
      
    \chapter{Random Parameter Filtering}
      In dit hoofdstuk volgt de informatie over het RPF algoritme zelf dat gebaseerd is op de paper en het technische report over RPF.
      \par
      Ook de instructieve resultaten komen in dit hoofdstuk, dit zijn de resultaten waarin duidelijk wordt waarom elke feature nuttig is bij het algoritme.
      
      
    \chapter{Resultaten}
      In dit hoofdstuk wordt besproken hoe het algoritme geimplementeerd is, 
      hoe de resultaten gerenderd zijn en wat het verschil is tussen het rpf algoritme gepresenteerd in de paper (snelheid, HDR, (features?)) en dit rpf algoritme.
    
    \chapter{Discussie}
      Vergelijk de resultaten met de resultaten uit het rpf paper en resultaten van andere papers (door middel van MSE bvb).
      Ook toekomstig werk wordt hier besproken.
      
    \chapter{Besluit}
      Een algemeen besluit wordt aangereikt in dit hoofdstuk.
    
    
    

    
    

  \end{document}