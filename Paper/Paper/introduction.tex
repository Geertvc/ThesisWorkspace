Computer Graphics has evolved from 2D to almost realistic 3D during the last decennia.
To reach this realism in graphics, Global Illumination algorithms became very important, 
these incorporate the indirect light caused by the surroundings.
These algorithms require large numbers of calculations and thus a large rendering time, especially when results without noise are expected.
To lower this rendering time, Monte Carlo methods are often used because they can be used to evaluate multidimensional functions in an unbiased way.
Complex and photorealistic images can be created using these Monte Carlo methods.
In the case of (Monte Carlo) rendering, the multidimensional integrals that needs to be solved at every pixel of the image are: 
integration of the radiance over the aperture of the camera (u,v), over the time the shutter is open ($t_0 \rightarrow t_1$) and over the area of the pixel ($(i-1/2,j-1/2) \rightarrow (i+1/2,j+1/2)$. 

\[
  I(i,j) = \int\limits_{i-\frac{1}{2}}^{i+\frac{1}{2}} 
		\int\limits_{j-\frac{1}{2}}^{j+\frac{1}{2}} 
		      \cdots
			  \int\limits_{-1}^{1} 
			      \int\limits_{-1}^{1}
				  \int\limits_{t_0}^{t_1} 
				    f(x,y,\cdots,u,v,t)
				  \,dt
			      \,dv
			  \,du
		\,dy
	    \,dx.
\]

As Monte Carlo methods take random samples to evaluate a function, a lot of samples are usually necessary to evaluate the function precisely and 
thus a lot of noise will be present when using a low number of samples. 
One obvious solution to this problem is to use more samples, 
but as the rendering time increases dramatically with the amount of samples this is not always an option.
Adaptive sampling algorithms can then be used to distribute these (sometimes large) numbers of samples in the best possible way across the image.
Reconstruction techniques can also be used to suppress the number of samples, 
as these try to use the available samples as much as possible, across pixels and even across frames.
Another technique that can be used is the focus of this paper and is an advanced image filter, 
it uses not only the color computed at each sample of the image, 
but also other parameters that are available during the rendering process, to filter the image.
\\
In the related work section I will describe all the different related papers I researched before deciding to research the RPF algorithm in more detail.
The technical sections describe the concept of bilateral filter after which the RPF algorithm is explained in detail.
To conclude the results describe my current results followed by a discussion of these results and a concise conclusion.
