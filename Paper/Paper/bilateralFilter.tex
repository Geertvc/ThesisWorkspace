The Bilateral filter is introduced in ~\cite{710815}, I have used some extra information from ~\cite{bilfilter} as well.
\\
As the name already suggests, a bilateral filter filters an image using two different inputs.
To understand the working of a bilateral filter, the concept of a Gaussian filter is explained first.

\subsection{Gaussian filter}
As an image has two dimensions, the 2D version of the gaussian filter will be explained, in fact this is simply the product of two one dimensional gaussians, one per direction.
The equation of a 2D Gaussian filter is defined as:

\begin{equation}
   g(x,y) = \frac{1}{2\pi\sigma^2}e^{-\frac{x^{2}+y^{2}}{2\sigma^{2}}}
\end{equation}

with x and y being the distance from the origin and $\sigma$ being the standard deviation of the Gaussian distribution.
The shape of a Gaussian filter being a 2D bell shaped function.
\\
When an image is filtered using a Gaussian filter, this will lead to blurred images as the values of each pixel will fade into the values of the surrounding pixels.
\todo{Add a Gaussian filtered image!!}

\subsection{Bilateral filter as two Gaussian filters}
A bilateral filter can be explained as being two Gaussian filter that filter respectively in the spatial (the domain filter D) and in the intensity domain (the range filter R).

\begin{equation}
  \begin{aligned}
    D(x,y) &=& e^{-\frac{x^{2}+y^{2}}{2\sigma^{2}}} \\
    R(p_i, p_j) &=& e^{-\frac{f(p_i)^{2}+f(p_j)^{2}}{2\sigma^{2}}}
  \end{aligned}
\end{equation}

The domain filter filters using the difference x and y between the positions of the two pixels and the range filter uses the intensity value (obtained by the image function $f$) of both pixels.
\\
We can now write down the equation to compute final value of a pixel filtered using a bilateral filter.

\begin{equation}
\begin{aligned}
   b(p_i) &=& k^{-1}\sum_{m=0}^{N-1} f(p_m)D(x,y)R(p_i,p_m) \\
   \text{with} \\
   k &=& \sum_{m=0}^{N-1} D(x,y)R(p_i,p_m)
\end{aligned}
\end{equation}

x and y are again the difference in the positions of the pixels $p_i$ and $p_m$ and N is the size of the neighboorhood taken into account for each pixel.


\subsection{Cross Bilateral Filter}
The RPF algorithm that is described in the following section, is based on a more advanced version of a bilateral filter, called the cross bilateral filter.
A Cross bilateral filter does not only filter in the spatial and the intensity domain but adds another filtering function to the filter.
This added function is a function just like the image function and returns a value for each position in the image.
Examples of this function are the world position at the first intersection point of the ray traced through that pixel, other features like the normal can also be chosen.
